% Template for Carleton student papers
% Author: Andrew Gainer-Dewar, 2013
% This work is licensed under the Creative Commons Attribution 4.0 International License.
% To view a copy of this license, visit http://creativecommons.org/licenses/by/4.0/ or send a letter to Creative Commons, 444 Castro Street, Suite 900, Mountain View, California, 94041, USA.
\documentclass[twoside]{article}
\usepackage{ccpaper}

% The Latin Modern font is a modernized replacement for the classic
% Computer Modern. Feel free to replace this with a different font package.
\usepackage{lmodern}

% Load in biblatex
% To use a different bibliography style, just change "numeric" to
% your preferred style (mla for MLA style, alphabetic for Author-Year
% style, etc.) There are a lot of options; check the BibLaTeX documentation.
\usepackage[backend=bibtex,style=numeric]{biblatex}
% Select the bibliography file
\addbibresource{references.bib}

\title{AI, Big Data, Bias, and Accountability}
\subtitle{Ethical Considerations and Antitrust of Big Tech} %Optional. Omit if not wanted.
\author{Emily Nomura}
\date{Spring 2021, Revised Winter 2024}
\course{Data, Ethics, and Society}
\prof{DATA 0080}

% To enable double spacing, uncomment this line:
%\doublespacing

\begin{document}
\maketitle{}

The simultaneous rise of big data and big tech in the twenty-first century is no coincidence. Our reliance on technology for everyday activities such as waking up or going to the grocery store is governed by the symbiotic relationship between big data and big tech. As “The Big Four” gained immense power and control in the technology industry, their anticompetitive and unethical practices have violated the outdated antitrust laws that the United States has in place. This paper analyzes and discusses the exponential growth and success of two of the most infamous companies in big tech, Amazon and Facebook (Meta). Although the significant advances in data modeling have provided hopeful and inspiring opportunities to solve the world’s most difficult problems, big data remains one of the main reasons for the daunting dominance of big tech their anticompetitive practices. 

Before diving into the specifics of big tech, big data, and antitrust, it is important to provide clear definitions and context. Big tech, also known as “The Big Four,” often refers to the technological powerhouses Google, Amazon, Facebook (Meta), and Apple. These companies have transformed the state of the internet economy. They have given billions of people around the world access to an enormous amount of information. However, they also receive heavy scrutiny about how they deal with user privacy, misinformation, bias, and anticompetitiveness \cite{antitrust}. Big tech companies  collect vast amounts of data to sell to third-parties and analyze consumer behavior.

Big data is most often associated with datasets that are simply too large or complex to be processed in a normal fashion. Some typical examples include social media feeds, health records, and genetic data. While most people are aware that their data is being collected and  sold by the social media sites they use, some are surprised to learn that taking a 23andMe genetic test could result in a rejected life insurance application \cite{genetic}. It is predicted that 60\% of white Americans can be identified through online genealogy databases and soon, 90\% of Americans of European descent will be able to be identified despite not submitting their own DNA to the genealogy companies themselves \cite{napkin}. The rise of big data is directly related to the dominance of big tech companies and their ability to collect, sell, and use customer information to their advantage.

The concept of antitrust comes from the Sherman Act of 1890, a law described as a “comprehensive charter of economic liberty aimed at preserving free and unfettered competition as the rule of trade'' \cite{ftc}. The three main objectives of the law were to have businesses operate more efficiently, keep prices down, and maintain quality. These same objectives from 1890 persist and are reflected in current antitrust laws. Business arrangements among competitors meant to fix prices, divide markets, or rig bids are clear violations of antitrust laws. The consequences of violating the Sherman Act or its counterpart, the Federal Trade Commission (FTC) Act, can be severe, maxing out at a one-hundred million dollar fine for businesses, a one million dollar fine for individuals, and even up to 10 years in prison \cite{ftc}. Despite these monetary fines, big tech companies often do not bat an eye when they are investigated for antitrust violations. 

Once a rudimentary online bookstore, Amazon has transformed into arguably the world's most dominant shopping platform. Since its origins in 1995, the business has exponentially expanded its operations and now holds power in a wide range of sectors, including healthcare, live-streaming services, and even the U.S. federal government. Although the website's capabilities were limited at launch, it still collected a great deal of user data. Amazon tracked each item that the customer viewed, what they put in their cart, their purchase history, what time they logged in, how long they spent on the site, and more \cite{amazon1}. As Amazon grew in popularity, the user-specific datasets they compiled became increasingly large and difficult to manage. The company used this data to study customer behavior and recommend certain books to their users to increase sales. Even today, users are greeted with images of recommended products based off of their purchase and search history from the moment they sign into their account. Amazon's intrusive data collection practices were not negatively viewed by employees at the time. They simply thought of it as “helping people make better decisions” as opposed to exploitative data collection \cite{amazon1}. In addition to their poor privacy practices, Amazon also avoids liability by having a purposefully long and confusing Terms and Conditions agreement. Most Amazon users do not know that when they create an account, they sign away their right to sue and hold the company accountable if they receive a product that is defective or dangerous \cite{amazon1}. This reluctance to be transparent about the types of data they collect and what rights customers have are just a few of Amazon's unethical practices.

Since its launch, Amazon has engaged in anticompetitive business strategies. Their current business plan bears similarities to their past methods to maximize profit - losing money in order to gain market share and drive other companies out of business \cite{amazon1}. Back when Amazon was a simple online bookstore, some publishers threatened to pull their books from the website because they wanted to keep a larger portion of their sales. In retaliation, Amazon would increase book prices and deliberately provide website links navigating customers to the publisher's ``worst competitor,'' thereby forcing vendors to comply with the company's strict regulations \cite{amazon1}. When Amazon first started delivering items to customers, Bezos and the rest of his executive team decided to create their own delivery system separate from FEDEX and UPS. The goal was for their trucks to be large enough to be stuffed full with Amazon packages but not so large that they would have to be regulated by the federal government \cite{amazon1}. If a crash involving one of their delivery trucks occurred, they simply claimed that the driver was a contract worker in order to avoid responsibility. Amazon’s dominance has allowed them to continuously abuse their power, violate antitrust laws with minimal punishment, and mistreat their workers.

It is common knowledge that Amazon prevents their warehouse employees from unionizing. In 2020, Amazon  workers said that they often felt disposable; if they didn't hit high enough packaging rates to maximize productivity, they were fired from their positions \cite{amazon1}. Many warehouse employees have submitted complaints to OSHA, claiming that the company does not prioritize worker’s health. This disregard for warehouse employees is a common theme that is even revealed in some of their marketing tactics. For example, a holiday-themed commercial from 2017 focused entirely on the shipment aspect at Amazon. It depicted animated, “happy” boxes singing “Give a Little Bit” together, hopping along conveyor belts, and practically shipping themselves to people around the world \cite{amazon2}. Like many other big tech companies, Amazon does not want their customers to acknowledge what goes on “behind the scenes.” Unfortunately, Amazon's business is not just limited to the online shopping industry. The rise of big data has provided enormous opportunity in computational and predictive fields, including advances in machine learning algorithms and artificial intelligence. Yet, big tech companies often do not disclose the true nature of their algorithms and the humans that help build them.

Amazon Web Services (AWS) provides data labeling assistance for companies that struggle to manage the amount of data they have, such as eBay \cite{aws}. In order for data labeling algorithms to ``learn'' and maximize both efficiency and accuracy, humans must first label the data. Amazon owns most of the major crowdsourcing services and employs data labelers \cite{ai}. The majority of data labeling offices are located in India where workers are promised a livable salary and a chance to improve their English skills \cite{ai}. While data labeling may seem like a simple task, some of the work involves watching medical operations, pornography, and other content that is extremely graphic or violent. Some workers even say that doctors should be doing the medical labeling themselves, as they have more domain-specific knowledge. However, data labeling is often thought of as a task that is “beneath” doctors. Unsurprisingly, other big tech companies share this sentiment. A Microsoft anthropologist claims that “the work is easily misunderstood. Listening to people cough all day may be disgusting, but that is also how doctors spend their days. We don’t think of that as drudgery” \cite{ai}. Not only is this statement incredibly tone deaf, but it also highlights the disconnect between higher up executives and their employees. The anthropologist’s words have a savior complex-like tone, comparing the data labelers to doctors when they are not at all regarded in the same light. Amazon's anticompetitive business strategies, deceptive data privacy policies, and mistreatment of its workers have landed it in hot water countless times in recent years. Meta, previously known as Facebook, is similarly accused of unethical practices that violate antitrust laws.\newpage

Meta's influence grew as technology and social media became more ubiquitous around the world. With billions of users creating accounts to connect with others, the company began to acquire smaller companies such as WhatsApp and Instagram, sparking antitrust concerns. The technological powerhouse has been under high scrutiny by American lawmakers calling for the breaking up of big tech companies \cite{facebook}. Meta became aware of these accusations and has strategically reduced the amount of acquisitions they are involved in to prevent inciting antitrust concerns. Meta has also begun to merge the operating systems of Messenger, WhatsApp, and Instagram \cite{facebook}. While some argue that it is just “smart business,” many others are wary of the real reasons behind the sudden change. Democratic Representative David Cicilline states, “The combination of Facebook, Instagram and WhatsApp into the single largest communications platform in history is a clear attempt to evade effective antitrust enforcement by making it harder for the company to be broken up” \cite{facebook}. These actions by Meta raise serious antitrust concerns, however, they rarely result in sufficient punishment.

Big data and its significant impact on the evolution of big tech companies introduces several ethical implications. Utilitarianism is an other-regarding ethical framework that is guided by the principle of utility - acting to produce the greatest good for the greatest number of people \cite{crashcourse2}. From a utilitarian perspective, the rise of big data and big tech may seem like a good thing. In fact, having big tech giants like Amazon and Meta dominate the technology industry may actually benefit society since services would be completely streamlined. In contrast, Kantian deontology deals with hypothetical imperatives, the desires of an individual, as well as categorical imperatives which are wholly dependent on moral obligations \cite{crashcourse1}. One of Kant's most popular formulations is the formula of humanity. This requires that one should not treat others as a mere means, but an end \cite{crashcourse1}. According to Kant, humans are not objects to be used or studied, rather, they are rational, autonomous beings. From a Kantian viewpoint, the very act of collecting user data with the goal of profiting off of ads would be considered unethical. An ideal Kantian society would prevent big tech companies from possessing so much power to ensure market equality and the universalizability principle, an ethical rule that states that one entity should not be allowed to make exceptions for themselves \cite{crashcourse1}.

Big tech companies like Amazon, Meta, Google, and Apple have been increasingly scrutinized for their violations of antitrust laws. A 2020 investigation by the United States Judiciary Antitrust Subcommittee concluded that “The Big Four” were “committed to drowning out competition through unfair and anti-competitive practices - often doing so as the expense of user privacy and innovation'' \cite{monopoly}. The mass collection of user data by big tech companies has been increasingly difficult to stray from in today’s age of consumerism. Technology is all around us; it has become interwoven in nearly every aspect of our lives. Amazon and Meta are two of the most prominent big tech companies with undeniable unethical and anticompetitive practices. Their market dominance has allowed them to avoid serious financial repercussions. As consumers, we should be given the option to control what kind of data is collected about us and hold companies accountable for infringements of data privacy and violations of antitrust.

\newpage
\printbibliography
\end{document}