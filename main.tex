% Template for Carleton student papers
% Author: Andrew Gainer-Dewar, 2013
% This work is licensed under the Creative Commons Attribution 4.0 International License.
% To view a copy of this license, visit http://creativecommons.org/licenses/by/4.0/ or send a letter to Creative Commons, 444 Castro Street, Suite 900, Mountain View, California, 94041, USA.
\documentclass[twoside]{article}
\usepackage{ccpaper}

% The Latin Modern font is a modernized replacement for the classic
% Computer Modern. Feel free to replace this with a different font package.
\usepackage{lmodern}

% Load in biblatex
% To use a different bibliography style, just change "numeric" to
% your preferred style (mla for MLA style, alphabetic for Author-Year
% style, etc.) There are a lot of options; check the BibLaTeX documentation.
\usepackage[backend=bibtex,style=numeric]{biblatex}
% Select the bibliography file
\addbibresource{references.bib}

\title{AI, Big Data, Bias, and Accountability}
\subtitle{Ethical Considerations and Antitrust of Big Tech} %Optional. Omit if not wanted.
\author{Emily Nomura}
\date{Spring 2021}
\course{Data, Ethics, and Society}
\prof{DATA 0080}

% To enable double spacing, uncomment this line:
%\doublespacing

\begin{document}
\maketitle{}

The simultaneous rise of big data and big tech in the twenty-first century is no coincidence. Our reliance on technology for everyday activities such as waking up or going to the grocery store is governed by the symbiotic relationship between big data and big tech. As “The Big Four” have gained immense power and control in the technology, shopping, electronics, and social media sectors, anticompetitive and unethical practices by these companies have violated the outdated antitrust laws that the United States has in place. The growth and success of the two most infamous companies when it comes to antitrust and big tech dominance, Amazon and Facebook, will be analyzed and discussed in-depth. Although the significant advances in big data-centered technologies such as machine learning and artificial intelligence have provided hopeful and inspiring opportunities to solve the world’s most difficult problems, it is clear that big data remains one of the main contributors to the daunting dominance of big tech and anticompetitive practices among the most powerful companies. 

Before diving into the specifics of big tech, big data, and antitrust, it is important to provide clear definitions and context. Big tech, also known as “The Big Four,” often refers to the technological powerhouses Google, Amazon, Facebook, and Apple. These companies have transformed the state of the internet economy and have given billions of people around the world access to an enormous amount of information. However, they also receive heavy scrutiny about how they deal with privacy, misinformation, bias, and anticompetitiveness \cite{antitrust}. Big tech companies use and collect vast amounts of data to not only sell, but also to analyze and predict information about their users and customers.

Big data is most often associated with datasets that are simply too large and/or complex to be processed and analyzed in a normal fashion. Some practical examples of big data include social media data, healthcare data, and, more obscurely, genetic data. While most people are aware that their data is being collected and sometimes sold by their healthcare providers and the social media sites they use, it can be surprising to learn that taking a 23andMe genetic test could potentially result in a rejected life insurance application \cite{genetic}. To give a better sense of the breadth of genetic data collected by American companies, it is predicted that currently, 60\% of white Americans can be identified through online genealogy databases and soon, 90\% of Americans of European descent will be able to be identified despite not submitting their own DNA to the genealogy companies \cite{napkin}. Thus, it is clear that the success of big tech companies is directly related to the rise of big data and said companies’ ability to collect, store, and use this information to their advantage.

The idea of antitrust arises from the Sherman Act of 1890, a law described as a “comprehensive charter of economic liberty aimed at preserving free and unfettered competition as the rule of trade" \cite{ftc}. The three main objectives of the law were to have businesses operate more efficiently, keep prices down, and keep quality up. These objectives are roughly the main reasons for the purpose of antitrust laws today. Business arrangements among competitors to fix prices, divide markets, or rig bids are clear violations. The consequences of violating the Sherman Act or its counterpart, the Federal Trade Commission (FTC) Act, can be severe, maxing out at a $100 million fine for businesses, a $1 million fine for individuals, and 10 years in prison \cite{ftc}. When questioning if two of the world’s largest companies, Amazon and Facebook, violate antitrust laws, there is much to discuss.

Amazon, once a simple online bookstore, has expanded in an almost exponential manner and now holds power in a wide range of sectors, including healthcare, the U.S. federal government, and even television, movie, and live streaming services. Even when Amazon was just starting out, they collected information about their users. For example, the rudimentary site tracked each item that the customer viewed, what they put in their cart, their purchase history, what time they logged in, how long they spent on the site, and much more \cite{amazon1}. As Amazon grew in popularity, the user-specific datasets that they compiled began to become increasingly large and difficult to work with, thus falling into the category of “big data.” Amazon used this information to study customer behavior and suggest books to their users to increase sales. This practice is clearly still employed today given the fact that users are greeted with recommended products based off of their purchase and search history as soon as they open the Amazon website. This seemingly intrusive data collection technique was not negatively viewed by employees at the time; they simply thought of it as “helping people make better decisions” rather than exploitative user data collection \cite{amazon1}. This invasion of privacy is just one of the anticompetitive and unethical actions that Amazon is repeatedly accused of.

Another anticompetitive practice that Amazon engages in centers around financial, business-oriented strategies. Just as Amazon began collecting and predicting user behavior back when the website was only meant for book buying, their previous financial strategy bears numerous similarities to some of their current methods meant to increase revenue. Back then, Amazon’s main approach was to lose money to gain market share in order to put other companies that couldn’t afford to lose money out of business \cite{amazon1}. An additional malicious tactic that Amazon used was meant to challenge publishers that threatened to pull their books from the website because Amazon was asking too high a portion of their sales. Amazon would change the prices of the books from these particular publishers up to list price and provided links that would navigate customers to their “worst competitor,” thus forcing vendors to comply with their requests and regulations \cite{amazon1}. Although these examples are some of Amazon’s earliest tactics that promote anticompetitiveness, it does not at all mean that they should go unpunished or ignored. In fact, because Amazon’s dominance has allowed them to continuously “get away” with methods that clearly violate antitrust laws, they are rarely ever punished or called out for other unethical practices they engage in, including avoiding liability and severely limiting workers’ rights.

The two most blatant ways in which Amazon avoids liability involves their delivery system and the nature of the Terms and Conditions agreement that users agree to when making an account. When Amazon first started delivering, Bezos and the rest of his executive team decided to create their own delivery system separate from FEDEX, UPS, USPS. The choice was less one of competition, rather, they aimed to hit a foolproof, malicious balance - they wanted their trucks to be large enough to be stuffed full with Amazon packages but not too large so they would have to be regulated by the federal government \cite{amazon1}. If a car crash occurs involving one of their delivery trucks, they simply claim that the driver is a contract worker, and therefore, avoids responsibility entirely. Amazon also avoids liability, as do most companies, by including a purposefully long, confusing Terms and Conditions section that the average user will not read. Many people do not know that when they click the “I Agree” button to create an Amazon account, they are signing away their rights to sue and hold the company accountable when purchasing a product that may be defective or dangerous \cite{amazon1}. Thus, as illustrated, Amazon’s history of ignored anticompetitive practices has allowed them to dominate the online shopping world and continue unethical behavior while avoiding liability.

One of Amazon’s most infamous qualities is their adamance to prevent their warehouse employees from unionizing. In previous employee interviews conducted in 2020, Amazon workers said that they felt disposable; if they didn’t hit high enough packaging rates to maximize productivity, they were kicked from their positions \cite{amazon1}. Many have heard of warehouse employees submitting complaints to OSHA, claiming that Amazon does not prioritize worker’s health. This disregard for their “low-ranked” employees is a common theme that is even evoked in their advertisements and commercials. For instance, a holiday-themed commercial from 2017 focused entirely on the shipment aspect of Amazon; it depicted animated, “happy” boxes singing “Give a Little Bit” together, hopping along conveyor belts, and practically shipping themselves to people around the world \cite{amazon2}. Amazon, like the rest of big tech companies, do not want people to think about what goes on “behind the scenes.” The rise of big data has provided enormous opportunity in computational and predictive fields, including advances in machine learning algorithms and artificial intelligence. However, big tech companies often do not disclose the true nature of the algorithms and the humans that help build them, which are often problematic and unethical.

A branch of Amazon called Amazon Web Services (AWS) provides data labelling assistance for companies that need help dealing with big data such as eBay \cite{aws}. What most people don’t know is that in order for data labelling algorithms to learn and maximize both efficiency and accuracy, humans must first label the data. Amazon owns most of the major crowdsourcing services and “employs” data labelers to help contribute to the model \cite{ai}. The majority of data labelling offices are located overseas in India where workers are given a liveable salary and a chance to improve their English skills \cite{ai}. While data labelling may seem like a somewhat simple task, part of the work may involve medical videos, pornography, or extremely graphic and violent content. Some workers even say that doctors and medical students should be doing the job themselves. The task is usually treated as “beneath” such highly renowned and respected individuals like doctors and med school students despite the fact that labelling accounts for 80\% of the time in building artificial intelligence technology \cite{ai}. It is obvious that data labellers are not respected or paid fairly by Amazon. Yet, other big tech companies often share the same dismissive and belittling sentiments as the online shopping giant. For example, a Microsoft anthropologist claims that “the work is easily misunderstood. Listening to people cough all day may be disgusting, but that is also how doctors spend their days. We don’t think of that as drudgery” \cite{ai}. Not only is this statement incredibly tone deaf, but it also highlights the fact that individuals in executive or administrative positions of power have no idea what the work is really like. The anthropologist’s words have an almost savior complex-like ring to them, directly comparing the data labellers to doctors when they are neither paid nor treated equally. Amazon and Microsoft are both incredibly powerful big tech companies that have grown alongside big data. Their resulting dominance and lack of competition allows them to disregard their workers’ rights and instead solely prioritize their profits and reputation.

Facebook is often regarded similarly to Amazon; that is, in a relatively negative light. Yet, news headlines focus less on the unethical practices and treatment of workers and more on their poor privacy practices and anticompetitive tactics. Facebook’s power expanded as technology and social media became more commonplace around the world. With millions upon millions of people signing up to use Facebook to connect with others, the company was able to collect a breadth of data about their user base, propelling them into the realm of big data. Thus, again, it is apparent that big data and big tech have a direct, symbiotic relationship. Two of Facebook’s most prominent anticompetitive practices involve the volatile realm of business acquisitions as well as technological specifications in their software. Both of these practices have been driven by the growth of big data and permitted due to the dominance of big tech companies.

After expanding into an empire with a multi-billion user base, Facebook has acquired many other notable tech companies and social media platforms like WhatsApp and Instagram. However, the technological powerhouse has been coming under high scrutiny by lawmakers such as Elizabeth Warren, who have been calling for the breaking up of big tech companies \cite{facebook}. One surprisingly blatant tactic that Facebook has been employing is significantly toning down the amount of acquisitions so as not to incite antitrust concerns. Still, the fact that they are limiting acquisitions so abruptly in order to curb scrutiny begs the question: Why does one, single company need so much power? Another more malicious anticompetitive practice that Facebook has begun to apply is merging the actual software regarding the messaging systems of Facebook Messenger, WhatsApp, and Instagram \cite{facebook}. While some argue that it is just “smart business,” many others are wary of the real reasons behind the sudden change. Democratic Representative David Cicilline states, “The combination of Facebook, Instagram and WhatsApp into the single largest communications platform in history is a clear attempt to evade effective antitrust enforcement by making it harder for the company to be broken up” \cite{facebook}. These actions by Facebook raise serious antitrust concerns, and have been greatly influenced by the growth of big data and big tech companies.

When discussing the effects of big data, the role of big tech, and the resulting antitrust concerns, it is important to consider the ethical implications that they introduce. Utilitarianism is an other-regarding ethical framework which is guided by the principle of utility - acting to produce the greatest good for the greatest number of people \cite{crashcourse2}. From a utilitarian perspective, the rise of big data and big tech wouldn’t necessarily be detrimental to society. In fact, having big tech giants like Amazon and Facebook dominate the technology industry and online shopping world may benefit us, since all services would be streamlined and controlled by the same companies. This is not the case for other ethical frameworks. Kantian deontology deals with hypothetical imperatives, the desires of an individual, as well as categorical imperatives, which are wholly dependent on moral obligations \cite{crashcourse1}. One of the most popular formulations of Kant is the formula of humanity, which requires that one should not treat others as a mere means, but an end \cite{crashcourse1}. This means that humans are not objects to be used or studied, rather, they are rational, autonomous beings. From a Kantian viewpoint, the mere collection of user-specific data with the goal of profiting off of ads or selling the data would be considered unethical. An ideal Kantian society would prevent big tech companies from possessing so much power not only because of their violations of antitrust and market equality, but also because of the universalizability principle, an ethical rule that states that one isn’t allowed to make exceptions for themselves. Thus, it is clear that the rise of big data, big tech, and antitrust practices is one that is, and will likely always be, difficult to evaluate with regards to the different ethical frameworks.

Big tech companies like Google, Amazon, Facebook, and Apple have been increasingly scrutinized for their suspected violations of antitrust laws. Indeed, a 2020 investigation by the United States Judiciary Antitrust Subcommittee concluded that “The Big Four” were “committed to drowning out competition through unfair and anti-competitive practices - often doing so as the expense of user privacy and innovation" \cite{monopoly}. The collection and use of big data by big tech companies has been increasingly difficult to stray from in today’s age of consumerism. Technology is all around us, and due to the COVID-19 pandemic, it has become interwoven with nearly every aspect of our lives, including education, shopping, and practically all forms of media consumption. Amazon and Facebook are two big tech companies that have implemented the most glaringly unethical practices. Their undeniable dominance has allowed them, for now, to avoid serious financial repercussions relating to antitrust concerns. It is difficult to assess the ethics of big data and big tech when there have been incredibly helpful outcomes guided by the technological advancements of the innovations of these companies. Nevertheless, as consumers, we should be given the option to control what kind of data is collected about us and be able to hold companies accountable on a micro level for infringements of data privacy and, on a broader level, for violations of antitrust.

\newpage

\printbibliography
\end{document}